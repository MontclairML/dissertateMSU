%% Includes all formatting for dissertateUSU (no longer relies on the DissertateUSU.cls file)
%% Added to RMarkdown by Tyson S. Barrett, PhD
%% Original Formatting by Sarah Schwartz, PhD

% Document class and options
\documentclass{DissertateUSU}
\usepackage[top=1in,bottom=1in,right=1in,left=1in]{geometry}

%% ==== %%%%%%%%%%%%%%%%% ==== %%
%% ==== %%   Header    %% ==== %%
%% ==== %%%%%%%%%%%%%%%%% ==== %%



\newlength{\cslhangindent}
\setlength{\cslhangindent}{1.5em}
\newenvironment{CSLReferences}%
  {}%
  {\par}


%%%%%%%%%%%%%%%%%%
%%  multipaper  %% -- works with monograph and n-paper designs
%%%%%%%%%%%%%%%%%%
% This is for the references (cite all at end or cite after each paper)
\usepackage[sectionbib]{natbib}
\usepackage{chapterbib}

%%%%%%%%%%%%%%%%%%%%%%%%
%% All Other Packages %%
%%%%%%%%%%%%%%%%%%%%%%%%
\usepackage{lastpage}
\usepackage{endnotes} % if authoryear option is used
\usepackage{graphicx} % Include figure files
\usepackage{dcolumn}  % Align table columns on decimal point
\usepackage{bm}       % bold math
\usepackage{amsmath,amsfonts} % popular packages from the American Mathematical Society
\usepackage{latexsym}         % latex symbols
\usepackage{lineno}           % for line numbers
\usepackage{array}            % for tables
\usepackage{ulem}             % for underlining
\usepackage{xcolor}           %  for color declarations
\usepackage{hyperref}         % for hypertext linking
% for the tightlist
\providecommand{\tightlist}{%
  \setlength{\itemsep}{0pt}\setlength{\parskip}{0pt}}
% For code highlighting and such
\usepackage{color}
\usepackage{fancyvrb}
\newcommand{\VerbBar}{|}
\newcommand{\VERB}{\Verb[commandchars=\\\{\}]}
\DefineVerbatimEnvironment{Highlighting}{Verbatim}{commandchars=\\\{\}}
% Add ',fontsize=\small' for more characters per line
\usepackage{framed}
\definecolor{shadecolor}{RGB}{248,248,248}
\newenvironment{Shaded}{\singlespacing\begin{snugshade}}{\end{snugshade}}
\newcommand{\AlertTok}[1]{\textcolor[rgb]{0.94,0.16,0.16}{#1}}
\newcommand{\AnnotationTok}[1]{\textcolor[rgb]{0.56,0.35,0.01}{\textbf{\textit{#1}}}}
\newcommand{\AttributeTok}[1]{\textcolor[rgb]{0.77,0.63,0.00}{#1}}
\newcommand{\BaseNTok}[1]{\textcolor[rgb]{0.00,0.00,0.81}{#1}}
\newcommand{\BuiltInTok}[1]{#1}
\newcommand{\CharTok}[1]{\textcolor[rgb]{0.31,0.60,0.02}{#1}}
\newcommand{\CommentTok}[1]{\textcolor[rgb]{0.56,0.35,0.01}{\textit{#1}}}
\newcommand{\CommentVarTok}[1]{\textcolor[rgb]{0.56,0.35,0.01}{\textbf{\textit{#1}}}}
\newcommand{\ConstantTok}[1]{\textcolor[rgb]{0.00,0.00,0.00}{#1}}
\newcommand{\ControlFlowTok}[1]{\textcolor[rgb]{0.13,0.29,0.53}{\textbf{#1}}}
\newcommand{\DataTypeTok}[1]{\textcolor[rgb]{0.13,0.29,0.53}{#1}}
\newcommand{\DecValTok}[1]{\textcolor[rgb]{0.00,0.00,0.81}{#1}}
\newcommand{\DocumentationTok}[1]{\textcolor[rgb]{0.56,0.35,0.01}{\textbf{\textit{#1}}}}
\newcommand{\ErrorTok}[1]{\textcolor[rgb]{0.64,0.00,0.00}{\textbf{#1}}}
\newcommand{\ExtensionTok}[1]{#1}
\newcommand{\FloatTok}[1]{\textcolor[rgb]{0.00,0.00,0.81}{#1}}
\newcommand{\FunctionTok}[1]{\textcolor[rgb]{0.00,0.00,0.00}{#1}}
\newcommand{\ImportTok}[1]{#1}
\newcommand{\InformationTok}[1]{\textcolor[rgb]{0.56,0.35,0.01}{\textbf{\textit{#1}}}}
\newcommand{\KeywordTok}[1]{\textcolor[rgb]{0.13,0.29,0.53}{\textbf{#1}}}
\newcommand{\NormalTok}[1]{#1}
\newcommand{\OperatorTok}[1]{\textcolor[rgb]{0.81,0.36,0.00}{\textbf{#1}}}
\newcommand{\OtherTok}[1]{\textcolor[rgb]{0.56,0.35,0.01}{#1}}
\newcommand{\PreprocessorTok}[1]{\textcolor[rgb]{0.56,0.35,0.01}{\textit{#1}}}
\newcommand{\RegionMarkerTok}[1]{#1}
\newcommand{\SpecialCharTok}[1]{\textcolor[rgb]{0.00,0.00,0.00}{#1}}
\newcommand{\SpecialStringTok}[1]{\textcolor[rgb]{0.31,0.60,0.02}{#1}}
\newcommand{\StringTok}[1]{\textcolor[rgb]{0.31,0.60,0.02}{#1}}
\newcommand{\VariableTok}[1]{\textcolor[rgb]{0.00,0.00,0.00}{#1}}
\newcommand{\VerbatimStringTok}[1]{\textcolor[rgb]{0.31,0.60,0.02}{#1}}
\newcommand{\WarningTok}[1]{\textcolor[rgb]{0.56,0.35,0.01}{\textbf{\textit{#1}}}}

% Tables
\usepackage{booktabs}
\usepackage{threeparttable}
\usepackage{array}
\newcolumntype{x}[1]{%
  >{\\centering\\arraybackslash}m{#1}}%
\usepackage{placeins}
\counterwithin{figure}{chapter}
\counterwithin{table}{chapter}
\usepackage[makeroom]{cancel}
%% Just for examples
\usepackage{lipsum}

%%%%%%%%%%%%%%%%%%%%%%%%%%%%%%%%%%%%%%%
%% Figure numbering - chapter.number %%
%%%%%%%%%%%%%%%%%%%%%%%%%%%%%%%%%%%%%%%
%\renewcommand{\thefigure}{\arabic{section}.\arabic{figure}}


%%%%%%%%%%%%%%%%%%%%%%%%%%%%%%%%%%%%%%%%%%%%%%%
%% LAYOUT: Title Page - info filled above    %%
%%%%%%%%%%%%%%%%%%%%%%%%%%%%%%%%%%%%%%%%%%%%%%%
\renewcommand{\maketitle}{
	\thispagestyle{empty}
	\vspace*{\fill}
	\begin{center}
	\doublespaced
	\MakeUppercase{COMPULSIVE TESTING AND ACADEMIC FUNCTIONING}\\
	by\\
	ME \\
	\singlespaced
	A dissertation submitted in partial fulfillment\\
	of the requirements for the degree \\
	\doublespaced
	of\\
	\MakeUppercase{IO Psych} \\
	in\\
	\singlespaced
  Field \\
	\end{center}

	\vspace{20pt}
	\noindent Approved: \\
	\vspace{30pt}
	\noindent
	\begin{tabular}{ll}
    \makebox[2.75in]{\hrulefill} & \makebox[2.75in]{\hrulefill}\\
    Chair Person                      & Committee Member 1 \\
    Major Professor              & Committee Member \\
    & \\
    & \\
    \makebox[2.75in]{\hrulefill} & \makebox[2.75in]{\hrulefill}\\
    Committee Member 2                 & Committee Member 3 \\
    Committee Member             & Committee Member \\
    & \\
    & \\
    \makebox[2.75in]{\hrulefill} & \makebox[2.75in]{\hrulefill}\\
    Committee Member 4                 & Richard S. Inouye, Ph.D. \\
    Committee Member             & Vice Provost for Graduate Studies \\

    \end{tabular}

  \vspace{20pt}
    \begin{center}
	  \singlespacing
      \MakeUppercase{Montclair State University}\\
	    Montclair, NJ\\
	    \doublespacing
	    2024
	  \end{center}
	\vspace*{\fill}
	\clearpage
}


% Abstract header
\newcommand{\abstracttitle}{

  \doublespacing
  \begin{center}
  COMPULSIVE TESTING AND ACADEMIC FUNCTIONING \\
  \vspace{12pt}
  by \\
  \vspace{12pt}
  ME \\
  Montclair State University, 2024
  \end{center}

  \vspace{12pt}

  \singlespacing
  \noindent Major Professor: Chair Person \\
  \noindent Department: Psychology

  \vspace{12pt}
}

%%%%%%%%%%%%%%%%%%%%%%%%%%%%%%%%%%%%%%%%%%%%%%%
%% LAYOUT: Copy Right - info filled above    %%
%%%%%%%%%%%%%%%%%%%%%%%%%%%%%%%%%%%%%%%%%%%%%%%
\newcommand{\copyrightpage}{
	\vspace*{\fill}
  \begin{center}
	\doublespacing
	Copyright \hspace{3pt}
	  \scshape \small \copyright  \hspace{3pt}
	  ME \hspace{3pt} 2024 \\
	All Rights Reserved
  \end{center}
	\vspace*{\fill}
}

\normalem



\begin{document}

\maketitle

\pagenumbering{roman}
\pagestyle{empty}
\copyrightpage

%% ==== %%%%%%%%%%%%%%%%% ==== %%
%% ==== %%    Body     %% ==== %%
%% ==== %%%%%%%%%%%%%%%%% ==== %%
\newpage

\pagestyle{plain} \fancyhead[R]{\thepage} \fancyfoot[C]{}

\chapter*{ABSTRACT}
\addcontentsline{toc}{section}{Abstract}

\abstracttitle
\doublespacing

Your abstract words go here. The line below puts the total number of
pages at the end of your abstract (as required).

\begin{flushright}(\pageref{LastPage} pages)\end{flushright}

\newpage
\fancyhead[R]{\thepage}
\fancyfoot[C]{}
\chapter*{PUBLIC ABSTRACT}
\addcontentsline{toc}{section}{Public Abstract}
\doublespacing

Your publicly abstracted words go here.

\newpage
\fancyhead[R]{\thepage}
\fancyfoot[C]{}
\chapter*{DEDICATION}
\addcontentsline{toc}{section}{Dedication}

Dedicate it.

\newpage
\fancyhead[R]{\thepage}
\fancyfoot[C]{}
\chapter*{ACKNOWLEDGMENTS}
\addcontentsline{toc}{section}{Acknowledgments}

Acknowledge those acknowledgable individuals and things.

\newpage
\fancyhead[R]{\thepage}
\fancyfoot[C]{}
\tableofcontents

\newpage
\fancyhead[R]{\thepage}
\fancyfoot[C]{}
\listoftables

\newpage
\fancyhead[R]{\thepage}
\fancyfoot[C]{}
\listoffigures

\newpage
\pagenumbering{arabic}

\newpage
\fancyhead[R]{\thepage}
\fancyfoot[C]{}

\chapter{DIEGO}

\doublespacing

Write your Chapter 1 in this file. Generally this chapter is the
introduction. We may have several citations (Barrett \& Schwartz, 2019;
R Core Team, 2018). Go ahead and erase all the text in this chapter (and
the other chapter files, as they are just trying to fill up space for
the example). Just keep the
\texttt{\textbackslash{}\textbackslash{}doublespacing} at the beginning
of this document.

\lipsum

\FloatBarrier
\newpage
\fancyhead[R]{\thepage}
\fancyfoot[C]{}

\chapter{Chapter 2's Title}

\doublespacing

Write your Chapter 2 in this file. Go ahead and erase all the text in
this chapter (and the other chapter files, as they are just trying to
fill up space for the example)

\lipsum

\FloatBarrier
\newpage
\fancyhead[R]{\thepage}
\fancyfoot[C]{}

\chapter{Chapter 3's Title}

\doublespacing

Write your Chapter 3 in this file. Go ahead and erase all the text in
this chapter (and the other chapter files, as they are just trying to
fill up space for the example)

\lipsum

\FloatBarrier
\newpage
\fancyhead[R]{\thepage}
\fancyfoot[C]{}

\chapter{Chapter 4's Title}

\doublespacing

Write your Chapter 4 in this file. Go ahead and erase all the text in
this chapter (and the other chapter files, as they are just trying to
fill up space for the example)

\lipsum

\FloatBarrier
\newpage
\fancyhead[R]{\thepage}
\fancyfoot[C]{}

\chapter{Chapter 5's Title}

\doublespacing

Write your Chapter 5 in this file. Go ahead and erase all the text in
this chapter (and the other chapter files, as they are just trying to
fill up space for the example)

\lipsum

\FloatBarrier
\newpage
\fancyhead[R]{\thepage}
\fancyfoot[C]{}

\chapter*{REFERENCES}

\setlength{\parindent}{-0.5in}
\setlength{\leftskip}{0.4in}
\setlength{\parskip}{6pt}

\noindent

\hypertarget{refs}{}
\begin{CSLReferences}{1}{0}
\leavevmode\vadjust pre{\hypertarget{ref-dissertateUSU}{}}%
Barrett, T. S., \& Schwartz, S. (2019). \emph{dissertateUSU:
DissertateUSU}.

\leavevmode\vadjust pre{\hypertarget{ref-RCoreTeam}{}}%
R Core Team. (2018). \emph{R: A language and environment for statistical
computing}. R Foundation for Statistical Computing.
\url{https://www.R-project.org/}

\end{CSLReferences}

%% ==== %%%%%%%%%%%%%%%%%%% ==== %%
%% ==== %% Bib and After %% ==== %%
%% ==== %%%%%%%%%%%%%%%%%%% ==== %%



\end{document}
